% Options for packages loaded elsewhere
\PassOptionsToPackage{unicode}{hyperref}
\PassOptionsToPackage{hyphens}{url}
%
\documentclass[
]{article}
\usepackage{amsmath,amssymb}
\usepackage{iftex}
\ifPDFTeX
  \usepackage[T1]{fontenc}
  \usepackage[utf8]{inputenc}
  \usepackage{textcomp} % provide euro and other symbols
\else % if luatex or xetex
  \usepackage{unicode-math} % this also loads fontspec
  \defaultfontfeatures{Scale=MatchLowercase}
  \defaultfontfeatures[\rmfamily]{Ligatures=TeX,Scale=1}
\fi
\usepackage{lmodern}
\ifPDFTeX\else
  % xetex/luatex font selection
\fi
% Use upquote if available, for straight quotes in verbatim environments
\IfFileExists{upquote.sty}{\usepackage{upquote}}{}
\IfFileExists{microtype.sty}{% use microtype if available
  \usepackage[]{microtype}
  \UseMicrotypeSet[protrusion]{basicmath} % disable protrusion for tt fonts
}{}
\makeatletter
\@ifundefined{KOMAClassName}{% if non-KOMA class
  \IfFileExists{parskip.sty}{%
    \usepackage{parskip}
  }{% else
    \setlength{\parindent}{0pt}
    \setlength{\parskip}{6pt plus 2pt minus 1pt}}
}{% if KOMA class
  \KOMAoptions{parskip=half}}
\makeatother
\usepackage{xcolor}
\usepackage[margin=1in]{geometry}
\usepackage{longtable,booktabs,array}
\usepackage{calc} % for calculating minipage widths
% Correct order of tables after \paragraph or \subparagraph
\usepackage{etoolbox}
\makeatletter
\patchcmd\longtable{\par}{\if@noskipsec\mbox{}\fi\par}{}{}
\makeatother
% Allow footnotes in longtable head/foot
\IfFileExists{footnotehyper.sty}{\usepackage{footnotehyper}}{\usepackage{footnote}}
\makesavenoteenv{longtable}
\usepackage{graphicx}
\makeatletter
\def\maxwidth{\ifdim\Gin@nat@width>\linewidth\linewidth\else\Gin@nat@width\fi}
\def\maxheight{\ifdim\Gin@nat@height>\textheight\textheight\else\Gin@nat@height\fi}
\makeatother
% Scale images if necessary, so that they will not overflow the page
% margins by default, and it is still possible to overwrite the defaults
% using explicit options in \includegraphics[width, height, ...]{}
\setkeys{Gin}{width=\maxwidth,height=\maxheight,keepaspectratio}
% Set default figure placement to htbp
\makeatletter
\def\fps@figure{htbp}
\makeatother
\setlength{\emergencystretch}{3em} % prevent overfull lines
\providecommand{\tightlist}{%
  \setlength{\itemsep}{0pt}\setlength{\parskip}{0pt}}
\setcounter{secnumdepth}{-\maxdimen} % remove section numbering
\ifLuaTeX
  \usepackage{selnolig}  % disable illegal ligatures
\fi
\usepackage{bookmark}
\IfFileExists{xurl.sty}{\usepackage{xurl}}{} % add URL line breaks if available
\urlstyle{same}
\hypersetup{
  pdftitle={STAT 216: Introduction to Statistics},
  pdfauthor={Syllabus for Online Sections: Summer 2024},
  hidelinks,
  pdfcreator={LaTeX via pandoc}}

\title{STAT 216: Introduction to Statistics}
\author{Syllabus for Online Sections: Summer 2024}
\date{}

\begin{document}
\maketitle

{
\setcounter{tocdepth}{3}
\tableofcontents
}
\begin{center}\rule{0.5\linewidth}{0.5pt}\end{center}

\section{Instructor contact
information}\label{instructor-contact-information}

Your primary contact in STAT 216 is your instructor. If you have
concerns that cannot be answered by your instructor, you may reach out
to the Student Success Coordinator.

Refer to your section's \emph{D2L Content: Instructor Contact
Information} page for further information on contacting your section
instructor.

\begin{center}\rule{0.5\linewidth}{0.5pt}\end{center}

\subsection{Student Success
Coordinator}\label{student-success-coordinator}

\href{https://www.math.montana.edu/directory/faculty/1582830/melinda-yager}{\textbf{Melinda
Yager}}\\
email:
\href{mailto:melinda.yager@montana.edu}{\nolinkurl{melinda.yager@montana.edu}}

\begin{center}\rule{0.5\linewidth}{0.5pt}\end{center}

\subsection{Session I: Sections 801 and 802
Instructors}\label{session-i-sections-801-and-802-instructors}

\textbf{Bernard Ntiamoah}\\
email:
\href{mailto:bernard.ntiamoah@student.montana.edu}{\nolinkurl{bernard.ntiamoah@student.montana.edu}}

\textbf{Nana Allotey}\\
email:
\href{mailto:nanaamaoyeallotey@montana.edu}{\nolinkurl{nanaamaoyeallotey@montana.edu}}

\textbf{Sarah Mensah}\\
email:
\href{mailto:sarahmensah@montana.edu}{\nolinkurl{sarahmensah@montana.edu}}

\textbf{Lakviru Perera}\\
email:
\href{mailto:korathotagelaperera@montana.edu}{\nolinkurl{korathotagelaperera@montana.edu}}

Grader: \textbf{McBeth Ahorter}\\
email:
\href{mailto:mcbethahortor@montana.edu}{\nolinkurl{mcbethahortor@montana.edu}}

\begin{center}\rule{0.5\linewidth}{0.5pt}\end{center}

\subsection{Session II: Section 803, 804, 805, and 806
Instructors}\label{session-ii-section-803-804-805-and-806-instructors}

\textbf{Melinda Yager}\\
email:
\href{mailto:melinda.yager@montana.edu}{\nolinkurl{melinda.yager@montana.edu}}

\textbf{Joseph Niakoh}\\
email:
\href{mailto:josephniakoh@montana.edu}{\nolinkurl{josephniakoh@montana.edu}}

\textbf{Benjamin Devries}\\
email:
\href{mailto:benjamindevries1@montana.edu}{\nolinkurl{benjamindevries1@montana.edu}}

\textbf{Ebenezer Mensah}\\
email:
\href{mailto:ebenezermensah1@montana.edu}{\nolinkurl{ebenezermensah1@montana.edu}}

\textbf{Madison Alderman}\\
email:
\href{mailto:madisonalderman1@montana.edu}{\nolinkurl{madisonalderman1@montana.edu}}

\section{Course calendars}\label{course-calendars}

\begin{itemize}
\tightlist
\item
  STAT 216 calendar for students in online sections\ldots{}

  \begin{itemize}
  \tightlist
  \item
    \href{calendars/Sum1_24-Stat216_Calendar-Online.pdf}{First Summer
    Session}
  \item
    \href{calendars/Sum2_24-Stat216_Calendar-Online.pdf}{Second Summer
    Session}
  \end{itemize}
\item
  \href{https://www.montana.edu/registrar/add_drop_schedule.html}{Summer
  registration calendar}
\end{itemize}

\begin{center}\rule{0.5\linewidth}{0.5pt}\end{center}

\section{Course description}\label{course-description}

Stat 216 is designed to engage you in the statistical investigation
process from developing a research question and data collection methods
to analyzing and communicating results. This course introduces basic
descriptive and inferential statistics using both traditional (normal
and \(t\)-distribution) and simulation approaches including confidence
intervals and hypothesis testing on means (one-sample, two-sample,
paired), proportions (one-sample, two-sample), regression and
correlation. You will be exposed to numerous examples of real-world
applications of statistics that are designed to help you develop a
conceptual understanding of statistics. After taking this course, you
should be able to:

\begin{itemize}
\tightlist
\item
  Understand and appreciate how statistics affects your daily life and
  the fundamental role of statistics in all disciplines.
\item
  Evaluate statistics and statistical studies you encounter in your
  other courses.
\item
  Critically read news stories based on statistical studies as an
  informed consumer of data.
\item
  Assess the role of randomness and variability in different contexts.
\item
  Use basic methods to conduct and analyze statistical studies using
  statistical software.
\item
  Evaluate and communicate answers to the four pillars of statistical
  inference: How strong is the evidence of an effect? What is the size
  of the effect? How broadly do the conclusions apply? Can we say what
  caused the observed difference?
\end{itemize}

\subsubsection{MUS STAT 216 learning
outcomes}\label{mus-stat-216-learning-outcomes}

\begin{enumerate}
\def\labelenumi{\arabic{enumi}.}
\tightlist
\item
  Understand how to describe the characteristics of a distribution.
\item
  Understand how data can be collected, and how data collection dictates
  the choice of statistical method and appropriate statistical
  inference.
\item
  Interpret and communicate the outcomes of estimation and hypothesis
  tests in the context of a problem.
\item
  To understand the scope of inference for a given dataset.
\end{enumerate}

\subsubsection{CORE 2.0}\label{core-2.0}

This course fulfills the Quantitative Reasoning (Q) CORE 2.0 requirement
because learning probability and statistics allows us to disentangle
what's really happening in nature from ``noise'' inherent in data
collection. It allows us to evaluate claims from advertisements and
results of polls and builds critical thinking skills which form the
basis of statistical inference. Students completing a Core 2.0
Quantitative Reasoning (Q) course should demonstrate an ability to:

Interpret and draw inferences from mathematical models such as formulas,
graphs, diagrams or tables. Represent mathematical information
numerically, symbolically and visually. Employ quantitative methods in
symbolic systems such as, arithmetic, algebra, or geometry to solve
problems.

Back to top

\begin{center}\rule{0.5\linewidth}{0.5pt}\end{center}

\section{Prerequisites}\label{prerequisites}

Entrance to STAT 216 requires at least one of the following be met:

\begin{itemize}
\tightlist
\item
  Grade of C- or better in a 100-level math course (or equivalent)
\item
  Grade of B or better in MATH 096
\item
  Level 30 on the \href{http://www.montana.edu/testing/MPLEX.html}{Math
  Placement Exam} or a combination of a good score on Math portion of
  SAT (540 or higher) or ACT (23 or higher) and/or good high school GPA

  \begin{itemize}
  \tightlist
  \item
    See the
    \href{http://www.math.montana.edu/undergrad/documents/MHiearchyFlowchart.pdf}{Math
    Prerequisite Flowchart} for more details.
  \end{itemize}
\end{itemize}

You should have familiarity with computers and technology (e.g.,
Internet browsing, word processing, opening/saving files, converting
files to PDF format, sending and receiving e-mail, etc.).

Back to top

\begin{center}\rule{0.5\linewidth}{0.5pt}\end{center}

\section{Course materials and
resources}\label{course-materials-and-resources}

\subsubsection{Online textbook and
coursepack}\label{online-textbook-and-coursepack}

Two ``textbooks'' are required for this course:

\begin{enumerate}
\def\labelenumi{\arabic{enumi}.}
\tightlist
\item
  \href{https://mtstateintrostats.github.io/IntroStatTextbook/}{\emph{Montana
  State Introductory Statistics with R}} --- our free, online textbook
\item
  {[}\emph{STAT 216 Coursepack}{]}--- available for free on D2L.
\end{enumerate}

\subsubsection{RStudio}\label{rstudio}

We will be using the statistical software
\href{https://www.r-project.org/}{R} through the IDE
\href{https://rstudio.com/products/rstudio/}{RStudio} for data
visualization and statistical analyses.

You will access this software through the MSU RStudio server:
\href{https://rstudio.math.montana.edu/}{rstudio.math.montana.edu}. Your
username is your 7-character NetID (in the form x\#\#x\#\#\#, where x is
a letter and \# is a number), and your password is the password
associated with your NetID. Your email address will not work to log in
to the RStudio server. \textbf{Note: Your NetID password expires every
180 days. To avoid losing access to RStudio server, please
\href{https://pwreset.montana.edu/react/}{reset your NetID password}
BEFORE the first day of your summer session}

See the
\href{https://mtstateintrostats.github.io/IntroStatTextbook/\#stat-computing}{Statistical
Computing} section in the Welcome chapter of our textbook for
alternative options for accessing RStudio.

\subsubsection{Learning management
tools}\label{learning-management-tools}

\begin{itemize}
\item
  \href{https://ecat1.montana.edu/}{\textbf{D2L}}: Find your instructor
  and co-instructor/TA contact info, announcements, exploration
  information, instructor notes, exam review material, assignment and
  data files, discussion forums, gradebook.

  \begin{itemize}
  \tightlist
  \item
    \emph{Important}: Make sure you are receiving email notifications
    for any D2L activity. In D2L, click on your name, then
    Notifications. Check that D2L is using an email address that you
    regularly check; you have the option of registering a mobile number.
    Check the boxes to get notifications for announcements, content,
    discussions, and grades.
  \item
    If you have a question about the course materials, computing, or
    logistics, please post your question to your D2L discussion board
    instead of emailing your instructor. This ensures all students can
    benefit from the responses. Other students are encouraged to
    respond.
  \end{itemize}
\item
  \href{https://www.gradescope.com/}{\textbf{Gradescope}}: Submit and
  review assignments, labs and exam grades. For more details, see our
  links in D2L under Content --\textgreater{} Primary Resources
  --\textgreater{} Gradescope Help
\item
  \href{https://math.montana.edu/undergrad/msc/}{\textbf{Math and Stat
  Center}}: Free drop-in tutoring for 100- and 200-level math and stat
  courses.
\end{itemize}

\begin{center}\rule{0.5\linewidth}{0.5pt}\end{center}

\section{Course format and
organization}\label{course-format-and-organization}

This is an online course. This perhaps obviously means we will never
meet in person. That greatly changes your role as a student as well as
ours as instructors.

\subsubsection{Weekly Expectations}\label{weekly-expectations}

Each week, online students will be expected to:

\begin{itemize}
\tightlist
\item
  \textbf{read} assigned sections of the textbook and \textbf{watch
  videos} on that week's content,
\item
  complete 1 - 4 \textbf{activities} in the coursepack and upload to
  \href{https://www.gradescope.com/}{Gradescope},
\item
  complete 2 \textbf{labs} in the coursepack and upload to
  \href{https://www.gradescope.com/}{Gradescope},
\item
  complete 2 \textbf{assignments} in
  \href{https://www.gradescope.com/}{Gradescope}.
\end{itemize}

\subsubsection{Online Instructor's Role}\label{online-instructors-role}

Most of the communication in the course should be done using the D2L
Discussion and Communication tools. Personal questions you may have can
be emailed to your instructor using the D2L Email link. Your instructor
will check email periodically, and try to respond to emails as promptly
as possible if the email is received during usual daytime or early
evening hours. Office hours will be the best way for live-streaming
Q\&A.

Questions on course material can be posted in the appropriate Discussion
folders for that topic. Responses from your instructor to those
questions will be posted as a response to that Discussion posting. You
are encouraged to post questions on course material in the Discussion
folder so that everyone may benefit from the responses! You can post
questions and respond to each other's questions. It honestly helps to
try to explain things to each other instead of waiting for the
instructor's responses, as trying to explain things out loud helps us
learn even if the answer is not correct! We all have a voice in an
online learning environment! If you wish to communicate with peers
within D2L to discuss homework questions and the like, any messages you
send privately to each other are not seen by course instructors. You may
even view the Classlist and email other members of the course to ask
questions or simply to generally communicate. Only what is posted in
Discussion Folders is available for everyone to see.

\subsubsection{Online Student's Role}\label{online-students-role}

It is important for you to understand that your role in this class is
not to be a passive note taker because there is no lecture in which to
take notes! Each of you is expected to read and understand the material
in the text, work on the homework problems, complete all assigned work,
and ask questions about concepts you do not understand (and hopefully
answer a few questions about the concepts you do understand) in the
discussion section of the course. Simply reading the words in the text
without comprehension, guessing your way through the assigned work, or
looking at the solutions to the homework and thinking you understand how
to do the problem is a waste of time and a sure way to fail the course.
Your instructor will answer your questions but you have to learn the
material on your own. In this class, more than any other course you have
taken, you are responsible for your education.

\emph{Note}: The proper way to learn material presented in this course
is to read the chapter, fill out the reading guide, watch the videos,
work through the activities, and then complete all assigned work on your
own. Don't rely only on the feedback to learn the material. Learning in
this way will not train you to think independently, and may show you how
to do one particular problem, but you will not be able to read another
problem and be able to solve it. Point being: statistics is about
understanding all aspects of an idea presented, not just a systematic
way of using a formula. In other words, this is NOT a math class!

\subsubsection{Group Expectations}\label{group-expectations}

Our use of groups in this course is based on educational research which
provides strong evidence that working in groups is effective and helps
us learn. Additionally, you will build team skills that employers look
for when hiring. By expressing your opinions and catching each other's
mistakes, you will learn to communicate statistical concepts. These are
partly ``common sense'' ideas (for instance, gathering more data
provides a better foundation for decision making), but they are often
phrased in odd ways. We find it really helps to talk about them with
others.

Back to top

\begin{center}\rule{0.5\linewidth}{0.5pt}\end{center}

\section{Course assessment}\label{course-assessment}

Your grade in STAT 216 will contain the following components.

\includegraphics{index_files/figure-latex/unnamed-chunk-1-1.pdf}

\subsubsection{Video quizzes (10\%)}\label{video-quizzes-10}

You will be expected to complete the assigned textbook reading and
videos prior to completing the activities(s) for each week. Each video
will discuss the reading content and walk you through one or more
examples for each section. Treat these videos as your lecture on the
textbook section, meaning you should actively be taking notes while
watching the videos. Concept check quiz questions will be completed in
Gradescope. \emph{(Note that you can rewind/review the video as much as
you like, and may submit your Video Quiz in Gradescope as many times as
you like up through the due date. You can also check your snwers in
Gradescope after submitting (prior to the due date) by clicking on each
question's hyperlink. If a explanation appears, that means you answered
the question correctly.)} All videos are linked in D2L and will
\emph{typically} be due each \textbf{Monday (or Tuesday) and Wednesday
at 11:59 pm MST}. Be sure to use the D2L checklists and course calendar
to determine the due date for each video.

\subsubsection{Activities (10\%)}\label{activities-10}

Activities are located within the Stat 216 coursepack. Activities will
be \emph{typically} due in
\href{https://www.gradescope.com/}{Gradescope} each \textbf{Tuesday and
Thursday at 8 pm MST}. Activities will be checked for completion. Be
sure to use the D2L checklists and course calendar to determine the due
date for each activity.

\subsubsection{Labs (15\%)}\label{labs-15}

Labs are located within the Stat 216 coursepack. Labs will
\emph{typically} be due in
\href{https://www.gradescope.com/}{Gradescope} each \textbf{Tuesday and
Thursday at 11:59 pm MST}. Labs should be used to check your
understanding of the activities within each module and will be graded
based on both completion and correctness. Be sure to use the D2L
checklists and course calendar to determine the due date for each lab.

\subsubsection{Assignments (15\%)}\label{assignments-15}

You will complete bi-weekly assignments in
\href{https://www.gradescope.com/}{Gradescope}. Assignments will
\emph{typically} be due each \textbf{Tuesday and Thursday at 11:59 pm
MST}. Assignments, like labs, will be used to assess your understanding
of each module and will be graded based on correctness. Be sure to use
the D2L checklists and course calendar to determine the due date for
each assignment.

\subsubsection{Midterm exams (20\%)}\label{midterm-exams-20}

There will be two midterm exams (worth 10\% each). Midterm 1 will take
place after completing modules 1, 2, and 3. Midterm 2 will take place
after completing modules 4, 5, 6, and 7. Midterm exams will be taken in
\href{https://www.gradescope.com/}{Gradescope}. You will have 3 hours
from the time you open the exam to complete it. \textbf{Exams must be
taken on the scheduled date. You are responsible for arranging your
schedule to accommodating taking of exams.} Be sure to use the D2L
checklists and course calendar to determine the due date for each
midterm.

\begin{itemize}
\tightlist
\item
  The midterm exams are intended as an opportunity for you to
  consolidate your knowledge from the first and second thirds of our
  course, so you are highly encouraged to take the exam using only your
  personal study materials. However, we rarely do statistics in a
  vacuum, so you will be allowed access to other students, the Math and
  Stat Center, and your instructor for help with questions. You are NOT
  allowed to use other online \tutoring~sites (such as Chegg).
\end{itemize}

\subsubsection{Final exam (15\%)}\label{final-exam-15}

The final exam will take place the final Thursday of the summer session.
It will be cumulative but focus more on modules 8, 9, and 10. The final
exam will be taken in \href{https://www.gradescope.com/}{Gradescope}.
You will have 3 hours from the time you open the exam to complete it.
\textbf{Exams must be taken on the scheduled date. You are responsible
for arranging your schedule to accommodating taking of exams.} Be sure
to use the D2L checklists and course calendar to determine the due date
for the final exam.

\begin{itemize}
\tightlist
\item
  The final exam is intended as an opportunity for you to consolidate
  your knowledge from the entire course, so you are highly encouraged to
  take the exam using only your personal study materials. However, we
  rarely do statistics in a vacuum, so you will be allowed access to
  other students, the Math and Stat Center, and your instructor for help
  with questions. You are NOT allowed to use other online
  \tutoring~sites (such as Chegg).
\end{itemize}

\subsubsection{Group Project (15\%)}\label{group-project-15}

The project project will be completed in parts working with 2-3 other
classmates. You will be required to submit a research question and data
collection plan, collect the data, and complete both a descriptive and
inferential analysis of the data. You will turn in one project component
per group each \textbf{Friday at 8 pm}.
\href{https://docs.google.com/document/d/1M-uz_NFvCF685WA9Gm4wZMDITWx1aRr5bYMWaMVGZNU/edit?usp=sharing}{Detailed
instructions, due dates, and rubrics for each project component are
available here.}

\emph{Plan ahead}: A 3-credit course over a 6 week summer session is
expected to account for 75 minutes each day, 5 days per week of
instructional time. This is the amount of time you should plan to spend
reading and taking notes on the textbook, watching and taking notes on
the videos, and attending synchronous learning sessions and office
hours. Additionally, our experience shows that an additional 15 to 22
hours per week of a 6 week course is required to obtain a good grade in
this class. By ``good'' we mean at least a C because a grade of D or
below does not count toward fulfilling degree requirements. Many of you
set your goals higher than just getting a C, and we fully support that.
You need roughly 20 to 25 hours per week to review past activities, read
feedback on previous assignments, complete current assignments, and
prepare for the next week's class.

\subsubsection{Late work policy}\label{late-work-policy}

In general, make-up exams or late homework assignments will not be
allowed. Case-by-case exceptions may be granted in only extreme cases at
the discretion of the instructor (daily work) or Student Success
coordinator (exams). You must speak to your instructor or Student
Success coordinator on or before the due date or exam date and provide
documentation explaining your absence for the instructor or Student
Success coordinator to determine whether an exception should be granted.
If you fail to provide documentation as requested then you will not be
able to make-up missed work at all.

\subsubsection{Letter grades}\label{letter-grades}

Final course grades will be determined according to the following scale.

\begin{longtable}[]{@{}ll@{}}
\toprule\noalign{}
Letter Grade & Weighted Score \\
\midrule\noalign{}
\endhead
\bottomrule\noalign{}
\endlastfoot
A & 93-100\% \\
A- & 90-92.99\% \\
B+ & 87-89.99\% \\
B & 83-86.99\% \\
B- & 80-82.99\% \\
C+ & 77-79.99\% \\
C & 70-76.99\% \\
D & 60-69.99\% \\
F & \textless59.99\% \\
\end{longtable}

The grade cutoffs may be shifted \emph{downward} at the end of the
semester based on student performance (never upward).

Back to top

\begin{center}\rule{0.5\linewidth}{0.5pt}\end{center}

\section{Diversity and inclusivity
statements}\label{diversity-and-inclusivity-statements}

Respect for Diversity: It is our intent that students from all diverse
backgrounds and perspectives be well-served by this course, that
students' learning needs be addressed both in and out of class, and that
the diversity that students bring to this class be viewed as a resource,
strength and benefit. It is our intent to present materials and
activities that are respectful of diversity: gender identity, sexual
orientation, disability, age, socioeconomic status, ethnicity, race,
religion, culture, perspective, and other background characteristics.
Your suggestions about how to improve the value of diversity in this
course are encouraged and appreciated. Please let us know ways to
improve the effectiveness of the course for you personally or for other
students or student groups.

In addition, in scheduling exams, we have attempted to avoid conflicts
with major religious holidays. If, however, we have inadvertently
scheduled an exam or major deadline that creates a conflict with your
religious observances, please let us know as soon as possible so that we
can make other arrangements.

Support for Inclusivity: We support an inclusive learning environment
where diversity and individual differences are understood, respected,
appreciated, and recognized as a source of strength. We expect that
students, faculty, administrators and staff at MSU will respect
differences and demonstrate diligence in understanding how other
peoples' perspectives, behaviors, and worldviews may be different from
their own.

Back to top

\begin{center}\rule{0.5\linewidth}{0.5pt}\end{center}

\section{Policy on academic
misconduct}\label{policy-on-academic-misconduct}

Students in an academic setting are responsible for approaching all
assignments with rigor, integrity, and in compliance with the University
Code of Student Conduct. This responsibility includes:

\begin{enumerate}
\def\labelenumi{\arabic{enumi}.}
\tightlist
\item
  consulting and analyzing sources that are relevant to the topic of
  inquiry;
\item
  clearly acknowledging when they draw from the ideas or the phrasing of
  those sources in their own writing;
\item
  learning and using appropriate citation conventions within the field
  in which they are studying; and
\item
  asking their instructor for guidance when they are uncertain of how to
  acknowledge the contributions of others in their thinking and writing.
\end{enumerate}

When students fail to adhere to these responsibilities, they may
intentionally or unintentionally ``use someone else's language, ideas,
or other original (not common-knowledge) material without properly
acknowledging its source'' \url{http://www.wpacouncil.org}. When the act
is intentional, the student has engaged in plagiarism.

Plagiarism is an act of academic misconduct, which carries with it
consequences including, but not limited to, receiving a course grade of
``F'' and a report to the Office of the Dean of Students. Unfortunately,
it is not always clear if the misuse of sources is intentional or
unintentional, which means that you may be accused of plagiarism even if
you do not intentionally plagiarize. If you have any questions regarding
use and citation of sources in your academic writing, you are
responsible for consulting with your instructor before the assignment
due date. In addition, you can work with an MSU Writing Center tutor at
any point in your writing process, including when you are integrating or
citing sources. You can make an appointment and find citation resources
at www.montana.edu/writingcenter.

\textbf{In STAT 216, students involved in plagiarism on assignments (all
parties involved) will receive a warning on the first offense and a 50\%
reduction in score. The second offense will result in a zero on that
assignment, and the incident will be reported to the Dean of Students.
Academic misconduct on an exam will result in a zero on that exam and
will be reported to the Dean of Students, without exception.}

\href{https://www.montana.edu/deanofstudents/academicmisconduct/academicmisconduct.html}{More
information about Academic Misconduct from the Dean of Students}

Back to top

\begin{center}\rule{0.5\linewidth}{0.5pt}\end{center}

\section{Copyright of intellectual
property}\label{copyright-of-intellectual-property}

This syllabus, course lectures and presentations, and any course
materials provided throughout this term are protected by U.S. copyright
laws. Students enrolled in the course may use them for their own
research and educational purposes. However, reproducing, selling or
otherwise distributing these materials without written permission of the
copyright owner is expressly prohibited, including providing materials
to commercial platforms such as Chegg or CourseHero. Doing so may
constitute a violation of U.S. copyright law as well as MSU's Code of
Student Conduct.

Back to top

\end{document}
